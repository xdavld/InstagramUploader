\chapter{Einleitung}
\label{sec:chapter1}

\textbf{Zielsetzung}\quad Das Ziel dieses Projekts ist die Entwicklung einer Webanwendung, die das tägliche Posten von Beiträgen auf Instagram automatisiert. Dabei sollen drei verschiedene 
Beitragstypen unterstützt werden:

\begin{itemize}
    \item Einzelbild-Posts
    \item Video-Posts
    \item Textbild-Posts
\end{itemize}

Jeder Beitrag soll mit passenden Hashtags versehen werden, um die Sichtbarkeit in sozialen Netzwerken zu erhöhen. Die Automatisierung soll über eine Weboberfläche 
gesteuert werden, sodass Nutzer die Beiträge zentral verwalten und planen können.

\textbf{Aufbau der Arbeit}\quad Die Arbeit gliedert sich in die folgenden Teile. \hyperref[sec:chapter1]{Kapitel 1} beschreibt die Zielsetzung des Projekts und legt den Grundstein für diese Arbeit.
In \hyperref[sec:chapter2]{Kapitel 2} wird die Konzeption erläutert, die das Fundament für die Entwicklung der Webanwendung bildet. \hyperref[sec:chapter3]{Kapitel 3} zeigt die Arbeitspakete 
auf und gibt eine Übersicht über die Beiträge der einzelnen Gruppenmitglieder. In \hyperref[sec:chapter4]{Kapitel 4} wird die Umsetzung detailliert beschrieben und eine Anleitung zum Starten der
Anwendung gegeben.
