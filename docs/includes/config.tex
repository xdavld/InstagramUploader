%%%%
%
% Zentrale Konfigurationsdatei
%
% In dieser Datei sind eine Reihe verpflichtender Einstellungen 
% (Nr. 1 bis 6) vorzunehmen.
%
% Die Einstellungen unter Nr. 7 bis 11 können im Regelfall unverändert
% belassen werden. Ausnahmen sind:
%  - Ihre Arbeit ist in englischer Sprache verfasst (Nr. 7)
%  - der Titel Ihrer Arbeit ist sehr lang, so dass er nicht auf das
%    Deckblatt passt oder anders umgebrochen werden soll (Nr. 8 und 9)
%  - es soll ein besonderes Abgabedatum angegeben werden (Nr. 10)
%  - Sie benötigen einen Vertraulichkeitsvermerk (Nr. 11)
%
%%%%


% TODO 1. Typ der Arbeit (für Titelseite und Metadaten)
% Zutreffendes auswählen:

%\newcommand{\typMeinerArbeit}{PA1} 
%\newcommand{\typMeinerArbeit}{PA2} 
\newcommand{\typMeinerArbeit}{Seminararbeit} 
%\newcommand{\typMeinerArbeit}{BA} 

% TODO 2. Vorname, Name der Autorin/des Autors (für Deckblatt und Metadaten)
\newcommand{\meinName}{Leonard Eckert, David Kreismann, Tobias Schnarr und Nico Wagner}

% TODO 3. Kurs eintragen
\newcommand{\meinKurs}{WWI2022F}

% TODO 4. Titel der Arbeit (für Deckblatt, ehrenwörtliche Erklärung und Metadaten, ohne Umbrüche angeben)
\newcommand{\themaMeinerArbeit}{Entwicklung einer Webanwendung zur automatisierten Veröffentlichung von Instagram-Beiträgen}



% OPTIONALE Einstellungen

% 7. Arbeit in Englisch
% (nur ändern, falls Ihre Arbeit in englischer Sprache geschrieben ist)
\newcommand{\meineSprache}{DE}	% Standard-Einstellung
% \newcommand{\meineSprache}{EN}	% für Arbeiten in englischer Sprache

% 8. Schriftgröße des Titels auf Deckblatt
% (nur ändern, falls Sie einen sehr langen Titel haben)
% Zutreffendes auswählen:
\newcommand{\schriftgroesseTitel}{\LARGE}   % Standard-Einstellung
%\newcommand{\schriftgroesseTitel}{\Large}  % bei sehr langen Titeln

% 9. Titel mit Umbrüchen für Deckblatt
% (nur ändern, falls Sie den Zeilenumbruch selbst beeinflussen möchten)
\newcommand{\titelAufDeckblatt}{\themaMeinerArbeit}		% Standard-Einstellung
%\newcommand{\titelAufDeckblatt}{Herausforderungen der Digitalisierung im globalen Wettbewerb von Industrieunternehmen \\ -- eine vergleichende Untersuchung unter Berücksichtigung aktueller und weniger aktueller Forschungsmethoden \\ am Beispiel der Firma Melanie Müller und Söhne AG} % explizite Angabe

% 10. Abgabedatum anpassen
% Zutreffendes auswählen:
\newcommand{\abgabeDatum}{\today}  		% Standard-Einstellung
%\newcommand{\abgabeDatum}{18.11.2024}  % falls nicht aktuelles Datum

% 11. Vertraulichkeitsvermerk
% (nur ändern, falls Ihre Arbeit einen Vertraulichkeitsvermerk tragen soll)
\newcommand{\hatVermerk}{nein}  	% Standard-Einstellung
%\newcommand{\hatVermerk}{ja}  	% falls Vertraulichkeitsvermerk
