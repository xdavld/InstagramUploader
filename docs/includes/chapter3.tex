\chapter{Arbeitspakete und Dokumentation der Beiträge}
\label{sec:chapter3}

\textbf{Arbeitspakete}\quad Zu Beginn des Projekts erfolgte eine allgemeine Aufteilung in zwei Teams. Team 1, bestehend aus David und Nico, war für die Entwicklung und das Aufsetzen des Frontends verantwortlich. Team 2, 
bestehend aus Leonard und Tobias, übernahm die Entwicklung des Backends.Die Aufteilung der Entwicklungsaufgaben in Arbeitspakete erfolgte anhand von Sprints. Im Rahmen dieses Projekts wurden insgesamt vier Sprints durchgeführt.
Im Folgenden werden die wichtigsten Ziele und wesentlichen Meilensteine der einzelnen Sprints dargestellt. 

\begin{enumerate}
    \item \textbf{Sprint 1} (03.12-06.12.2024)
    \begin{itemize}
        \item Das Ziel des ersten Sprints war die Definition der MUSS-, SOLL- und KANN-\\Anforderungen sowie die Entwicklung des Konzepts, das ausführlich in \hyperref[sec:chapter2]{Kapitel 2} beschrieben wurde. Außerdem sollte 
        ein erstes Frontend aufgesetzt und eine Interaktion mit der Instagram-\ac{API} ermöglicht werden. Ein weiteres Ziel war die Durchführung eines ersten Bild-Uploads, um die Verbindung zur \ac{API} zu testen.
        In diesem Sprint konnten folgende Erfolge erzielt werden:
        \begin{itemize}
            \item Bild-Upload 
            \item Frontend mit Drag \& Drop Support
            \item Verbindung zur Instagram-\ac{API}
        \end{itemize}
    \end{itemize}

    \item \textbf{Sprint 2} (06.12-13.12.2024)
    \begin{itemize}
        \item Das Ziel des zweiten Sprints war die Implementierung eines Backend-Logins, um Nutzern eine sichere Authentifizierung zu ermöglichen. Zudem sollte das Frontend um ein responsive Design erweitert werden, um die Anwendung 
        auf unterschiedlichen Geräten optimal nutzbar zu machen. Zudem wurde die Architektur der Anwendung weiter verfeinert. Die Ergebnisse dieses Sprints umfassen:
        \begin{itemize}
            \item Backend-Log-In
            \item Responsive Design
        \end{itemize}
    \end{itemize}

    \item \textbf{Sprint 3} (14.12-10.01.2025) 
    \begin{itemize}
        \item Der dritte Sprint hatte das Ziel, eine Datenbankanbindung herzustellen, um Benutzerdaten und Inhalte effizient zu speichern und abzurufen. In der Datenbank werden zentrale Informationen zu hochgeladenen Medien, 
        wie z.B. Medien-URLs, Erstellungszeitpunkte und Veröffentlichungsstatus, verwaltet. Darüber hinaus sollte die Möglichkeit des Video-Uploads implementiert werden, um die Funktionalität der Anwendung zu erweitern. 
        In diesem Sprint konnten folgende Erfolge erzielt werden:
        \begin{itemize}
            \item Datenbankanbindung
            \item Video-Upload
        \end{itemize}
    \end{itemize}

    \item \textbf{Sprint 4} (10.01-17.01.2025)
    \begin{itemize}
        \item Im vierten und letzten Sprint lag der Fokus auf der Implementierung eines Kalenders, der Nutzern ermöglicht, Veröffentlichungen zu planen und geplante Beiträge einzusehen. Zusätzlich wurde ein Scheduler integriert, 
        um geplante Veröffentlichungen automatisch durchzuführen. Schließlich wurde die OpenAI-\ac{API} angebunden, um KI-gestützte Funktionen wie das Generieren von Hashtags und Bildbeschreibungen bereitzustellen. Darüber hinaus 
        wurden verschiedene Tests implementiert, um die volle Funktionalität der Anwendung sicherzustellen. Im Rahmen dieses Sprints wurden folgende Ziele erreicht:
        \begin{itemize}
            \item Kalender
            \item Scheduler
            \item OpenAI-\ac{API}-Anbindung
            \item Tests
        \end{itemize}
    \end{itemize}
\end{enumerate}

Es ist zu beachten, dass die hier dargestellten Punkte nicht den vollständigen Umfang der umgesetzten Features und Funktionalitäten widerspiegeln. Sie stellen lediglich einen Auszug dar, enthalten jedoch alle MUSS-Anforderungen. 
Eine detaillierte Beschreibung der gesamten Anwendung erfolgt im nachfolgenden \hyperref[sec:chapter4]{Kapitel 4}.

\textbf{Dokumentation der Beiträge}\quad Die Teams und die Aufgabenteilung (Backend und Frontend) wurden während des gesamten Projektzeitraums beibehalten. Es ist zu beachten, dass häufig mit Pair Programming gearbeitet wurde, 
weshalb einzelne Personen deutlich weniger Commits in GitHub aufweisen. Die Dokumentation wurde von allen Mitgliedern zu gleichen Teilen erstellt.  