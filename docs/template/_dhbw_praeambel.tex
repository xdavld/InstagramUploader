%%% Präambel %%%
% hier sollten keine Änderungen erforderlich sein
%
\usepackage{ifthen}           % für Umschaltung DE/EN
\newcommand{\DEoEN}[2]{\ifthenelse{\equal{\meineSprache}{DE}}{#1}{#2}}

\usepackage[utf8]{inputenc}   % Zeichencodierung UTF-8 für Eingabe-Dateien
\usepackage[T1]{fontenc}      % Darstellung von Umlauten im PDF
\usepackage{amsmath} % for math
\usepackage{multicol} % for multiple columns
\usepackage{booktabs} % for table formation
\usepackage{caption} % for captions in the appendix
\renewcommand{\theequation}{\arabic{equation}} % disable chapter-based numbering
\usepackage{float} % for figures
\usepackage{array} % for tables

\usepackage{listings}         % für Einbindung von Code-Listings
\lstset{numbers=left,numberstyle=\tiny,numbersep=5pt,texcl=true}
\lstset{literate=             % erlaubt Sonderzeichen in Code-Listings 
{Ö}{{\"O}}1
{Ä}{{\"A}}1
{Ü}{{\"U}}1
{ß}{{\ss}}2
{ü}{{\"u}}1
{ä}{{\"a}}1
{ö}{{\"o}}1
{€}{{\euro}}1
}

\usepackage[
  inner=35mm,outer=15mm,top=25mm,
  bottom=20mm,foot=12mm,includefoot
]{geometry}                 % Einstellungen für Ränder

\DEoEN{
  \usepackage[ngerman]{babel} % Spracheinstellungen Deutsch
  \usepackage[babel,german=quotes]{csquotes} % deutsche Anf.zeichen
}{
 \usepackage[english]{babel} % Spracheinstellungen Englisch
 \usepackage[babel,english=british]{csquotes} % englische Anf.zeichen
}

\usepackage{enumerate}      % anpassbare Nummerier./Aufz.
\usepackage{graphicx}       % Einbinden von Grafiken
\usepackage[onehalfspacing]{setspace} % anderthalbzeilig

\usepackage{blindtext}      % Textgenerierung für Testzwecke
\usepackage{color}          % Verwendung von Farbe 

\usepackage{acronym}        % für ein Abkürzungsverzeichnis

\usepackage[                % Biblatex
  backend=biber,
  bibstyle=_dhbw_authoryear,maxbibnames=99,
  citestyle=authoryear,     
  uniquename=true, useprefix=true,
  bibencoding=utf8]{biblatex}
\input{template/_dhbw_biblatex-config.tex} % mit DHBW-spezifischen Einstellungen

\usepackage{hyperref}       % URL-Formatierung, klickbare Verweise

\usepackage{tocloft}        % für Verzeichnis der Anhänge

\usepackage{multirow}       % Tabellenformatierung 

\newcounter{anhcnt}
\setcounter{anhcnt}{0}
\newlistof{anhang}{app}{}

\newcommand{\anhang}[1]{%
  \refstepcounter{anhcnt}
  \setcounter{anhteilcnt}{0}
  \section*{\appendixname\ \theanhcnt: #1}
  \addcontentsline{app}{section}{\protect\numberline{\appendixname\ \theanhcnt}#1}\par
}

\newcounter{anhteilcnt}
\setcounter{anhteilcnt}{0}

\newcommand{\anhangteil}[1]{%
	\refstepcounter{anhteilcnt}
	\subsection*{\appendixname\ \arabic{anhcnt}/\arabic{anhteilcnt}: #1}
	\addcontentsline{app}{subsection}{\protect\numberline{\appendixname\ \theanhcnt/\arabic{anhteilcnt}}#1}\par
}

\renewcommand{\theanhteilcnt}{\appendixname\ \theanhcnt/\arabic{anhteilcnt}}

% vgl. S. 4 Paket-Beschreibung tocloft 	
% Einrückungen für Anhangverzeichnis
\makeatletter
\newcommand{\abstaendeanhangverzeichnis}{
\renewcommand*{\l@section}{\@dottedtocline{1}{0em}{5.5em}}
\renewcommand*{\l@subsection}{\@dottedtocline{2}{2.3em}{6.5em}}
}
\makeatother

% Einrückungen
\makeatletter
\renewcommand*{\l@figure}{\@dottedtocline{1}{0em}{2.3em}}
\renewcommand*{\l@table}{\@dottedtocline{1}{0em}{2.3em}}
\makeatother


\usepackage{chngcntr}                % fortlaufende Zähler für Fußnoten, Abbildungen und Tabellen
\counterwithout{figure}{chapter}
\counterwithout{table}{chapter}
\counterwithout{footnote}{chapter}

\usepackage[automark]{scrlayer-scrpage} 
\input{template/_dhbw_kopfzeilen.tex}		 % für schöne Kopfzeilen 

\usepackage{textcomp}            % erlaubt EUR-Zeichen in Eingabedatei
\usepackage{eurosym}             % offizielles EUR-Symbol in Ausgabe
\renewcommand{\texteuro}{\euro}  % ACHTUNG: nach hyperref aufrufen!

\usepackage{scrhack}             % stellt Kompatibilität zw. KOMA-Script
                                 % (scrreprt) und anderen Paketen her
                                 
% Anpassung der Abstände bei Kapitelüberschriften
% (betrifft auch Inhalts-, Abbildungs- und Tabellenverzeichnis)
\renewcommand*\chapterheadstartvskip{\vspace*{-\topskip}}
\newcommand{\myBeforeTitleSkip}{1mm}
\newcommand{\myAfterTitleSkip}{10mm}
\setlength\cftbeforetoctitleskip{\myBeforeTitleSkip}
\setlength\cftbeforeloftitleskip{\myBeforeTitleSkip}
\setlength\cftbeforelottitleskip{\myBeforeTitleSkip}

\setlength\cftaftertoctitleskip{\myAfterTitleSkip}
\setlength\cftafterloftitleskip{\myAfterTitleSkip}
\setlength\cftafterlottitleskip{\myAfterTitleSkip}

% Anhang beginnen
\newcommand{\startAnhang}{%
\chapter*{\appendixname}
\addcontentsline{toc}{chapter}{\appendixname}
\section*{\anhangVzBezeichnung}
\vspace{-8em}

% vor \listofanhang müssen Einrückungen angepasst werden
\abstaendeanhangverzeichnis
\spezialkopfzeile{\DEoEN{Anhang}{Appendix}} % damit in der Kopfzeile das Wort "Anhang" angezeigt wird
}

% Abkürzungsverzeichnis beginnen
\newcommand{\startAbkVerzeichnis}{%
\chapter*{\abkVzBezeichnung}
\addcontentsline{toc}{chapter}{\abkVzBezeichnung}
}

% Sprach-spezifische Einstellungen
\DEoEN{%
\newcommand{\abkVzBezeichnung}{Abkürzungsverzeichnis}
\newcommand{\anhangVzBezeichnung}{Anhangverzeichnis}

\renewcaptionname{ngerman}{\refname}{Literaturverzeichnis} % statt "Literatur"
\renewcaptionname{ngerman}{\figurename}{Abb.}
\renewcaptionname{ngerman}{\tablename}{Tab.}
}{
\newcommand{\abkVzBezeichnung}{Abbreviations}
\newcommand{\anhangVzBezeichnung}{Appendix directory}

\renewcaptionname{english}{\contentsname}{Table of Contents}
\renewcaptionname{english}{\figurename}{Fig.}
\renewcaptionname{english}{\tablename}{Tab.}
}


                                                            
%%% Ende der Präambel %%%