\clearpage

\thispagestyle{empty}
\DEoEN{%
{\LARGE\textsf{\textbf{Erklärung zur Verwendung generativer KI-Systeme}}\bigskip}

Bei der Erstellung der eingereichten Arbeit habe ich die nachfolgend aufgeführten auf künstlicher Intelligenz (KI) basierten Systeme benutzt:

\begin{enumerate}
\item ChatGPT
\end{enumerate}

Ich erkläre, dass ich

\begin{itemize}
  \item mich aktiv über die Leistungsfähigkeit und Beschränkungen der oben genannten 
KI-Systeme informiert habe,\footnote{U.a. gilt es hierbei zu beachten, dass an KI weitergegebene Inhalte ggf. als Trainingsdaten genutzt und wiederverwendet werden. Dies ist insb. für betriebliche Aspekte als kritisch einzustufen.}
  \item die aus den oben angegebenen KI-Systemen direkt oder sinngemäß übernommenen Passagen gekennzeichnet habe,
%
% In der Fußnote Ihrer Arbeit geben Sie die KI als Quelle an, z.B.: 
% Erzeugt durch Microsoft Copilot am dd.mm.yyyy. 
% Oder: Entnommen aus einem Dialog mit Perplexity vom dd.mm.yyyy. 
% Oder: Absatz 2.3 wurde durch ChatGPT sprachlich geglättet.
%
  \item überprüft habe, dass die mithilfe der oben genannten KI-Systeme generierten und von mir übernommenen Inhalte faktisch richtig sind,
  \item mir bewusst bin, dass ich als Autorin bzw. Autor dieser Arbeit die Verantwortung für die in ihr gemachten Angaben und Aussagen trage.
\end{itemize}

Die oben genannten KI-Systeme habe ich wie im Folgenden dargestellt eingesetzt: 


\begin{tabular}{|p{4cm}|p{3cm}|p{7cm}|}
    \hline
    \textbf{Arbeitsschritt in der wissenschaftlichen Arbeit} &
%
% Beispiele hierfür sind u.a. die folgenden Arbeitsschritte: 
% Generierung von Ideen, Konzeption der Arbeit, Literatursuche, Literaturanalyse, 
% Literaturverwaltung, Auswahl von Methoden, Datensammlung, Datenanalyse, 
% Generierung von Programmcodes
%
% Wenn Sie unsicher sind, ob Sie ein verwendetes KI-System angeben müssen, 
% wenden Sie sich an Ihre:n Betreuer:in.
%
    \textbf{Eingesetzte(s) KI-System(e)} & \textbf{Beschreibung der Verwendungsweise} \\
    \hline
    Korrektur der Arbeit & ChatGPT & Einzelne Kapitel ChatGPT zum Korrigieren gegeben. Erfolg: geringfügig, nach der Korrektur wurden noch einige Fehler gefunden.\\
    \hline
  \end{tabular}
} % Ende deutscher Teil
{% Beginn englische Erklaerung
{\LARGE\textsf{\textbf{Declaration on the Use of Generative AI Systems}}\bigskip}

In preparing the submitted work, I have used the following artificial intelligence (AI)-based systems:

\begin{enumerate}
\item
\item
\item \ldots
\end{enumerate}

I hereby declare that I

\begin{itemize}
  \item actively informed myself about the capabilities and limitations of the above-mentioned AI systems,\footnote{In particular, it should be noted that content passed on to AI may be used as training data and reused. This is to be considered critical, especially for operational aspects.}
  \item indicated passages directly or indirectly adopted from the above-mentioned AI systems,
%
% In the footnote of your work, cite the AI as the source, e.g.:
% Generated by Microsoft Copilot on mm.dd.yyyy. 
% Or: Taken from a dialogue with Perplexity on mm.dd.yyyy. 
% Or: Paragraph 2.3 was linguistically smoothed by ChatGPT.
%
  \item verified that the content generated and adopted by me using the above-mentioned AI systems is factually correct,
  \item am aware that as the author of this work, I bear responsibility for the statements and information provided in it.
\end{itemize}

I have utilized the above-mentioned AI systems as illustrated below: 

\begin{center}
\begin{tabular}{|p{4cm}|p{3cm}|p{7cm}|}
    \hline
    \textbf{Task in the scientific work} &
%
% Examples of such tasks include: idea generation, 
% conception of the work, literature search, literature analysis, 
% literature management, selection of methods, data collection, 
% data analysis, generation of program code
%
% If you are unsure whether you need to specify an AI system used, 
% consult your supervisor.
%
	 \textbf{AI System(s) Used} & \textbf{Description of Usage} \\
    \hline
    & & \\ % Placeholder for your entries, insert your information here
    \hline
    & & \\ % (Expand the table as needed and continue on subsequent pages)
    \hline
    & & \\
    \hline
    & & \\
    \hline
  \end{tabular}
\end{center}
}
